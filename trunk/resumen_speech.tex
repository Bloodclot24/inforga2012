% Tama\~no de letra.
\documentclass[12pt,titlepage]{article}

%------------------------------ Paquetes ----------------------------------

% Paquetes:

%Para comentarios multil\'inea.
\usepackage{verbatim}

% Para tener cabecera y pie de p\'agina con un estilo personalizado.
\usepackage{fancyhdr}

% Codificaci\'on UTF-8
\usepackage[utf8]{inputenc}

% Castellano.
\usepackage[spanish]{babel}

% Tama\~no de p\'agina y m\'argenes.
\usepackage[a4paper,headheight=16pt,scale={0.75,0.8},hoffset=0.5cm]{geometry}

% Para poder agregar notas al pie en tablas:
%\usepackage{threeparttable}

% Tipo de letra Helvetica (Arial).
%\usepackage{helvet}
%\renewcommand\familydefault{\sfdefault}

% Gr\'aficos:

% Para incluir im\'agenes, el siguiente c\'odigo carga el paquete graphicx
% seg\'un se est\'e generando un archivo dvi o un pdf (con pdflatex).

% Para generar dv.
%\usepackage[dvips]{graphicx}

% Para generar pdf.
\usepackage[pdftex]{graphicx}
\pdfcompresslevel=9

\usepackage{pdfpages}

%
% Directorio donde est\'an las imagenes.
%
%\newcommand{\imgdir}{includes}
%\graphicspath{{\imgdir/}}

%------------------------------ ~paquetes ---------------------------------

%------------------------- Inicio del documento ---------------------------
\begin{document}
% -------------------------- T\'itulo y autor(es) ---------------------------

% \'a \'e \'i \'o \'u \~n

\title{Grupo 2: IEP de Iluminaci\'on}
\author{}

\section{Selecci\'on de la empresa}
El grupo cuenta con contactos en tres empresas de producci\'on de bienes: Guaymall\'en, SAEMSA e IEP de Iluminaci\'on.

En el caso de Guaymall\'en, el tama\~no de la empresa es acorde a las necesidades del Trabajo Pr\'actico, y el contacto es de la confianza de uno de los integrantes del grupo, pero su ubicaci\'on en el barrio de Liniers es un inconveniente para la mayor\'ia de los integrantes del grupo, lo que dificultar\'ia las entrevistas.

En el caso de SAEMSA, si bien su tama\~no excede un poco los requerimientos de la c\'atedra se puede considerar aceptable. La ubicaci\'on no es un inconveniente en este caso pero el contacto, m\'as all\'a de ser confiable, no dispone de informaci\'on relevante a los efectos del Trabajo.

En el caso de IEP de Iluminaci\'on el tama\~no es correcto, el contacto es confiable y con alta disponibilidad de informaci\'on relevante. Si bien la ubicaci\'on es algo alejada, decidimos sortear este aspecto para tener tranquilidad respecto de la disponibilidad de informaci\'on, y por ello elegimos a IEP de Iluminaci\'on.

\section{Historia}
En sus inicios (IEP) Industrias Electrot\'ecnicas Puig  se especializaba en la fabricaci\'on de aparatos reflectores para alumbrado en la ciudad espa\~nola de Barcelona.

En el a\~no 1966 pasa a formar parte del Grupo Simon, un holding de empresas del mercado el\'ectrico espa\~nol con visionaria expansi\'on hacia a los cinco continentes.

A principios de la d\'ecada del 90 adapta nuevamente su estructura y su imagen, empezando a conoc\'ersela como IEP DE ILUMINACI\'ON.

Finalmente, es en el a\~no 1998 que IEP DE ILUMINACI\'ON llega a la Argentina, convirti\'endose en el lapso de 12 a\~nos en la principal responsable de la comercializaci\'on de luminarias para toda Am\'erica del Sur.

El Grupo Simon fue incorporando nuevos centros de producci\'on y actualmente su presencia mundial alcanza a m\'as de 50 pa\'ises, con sede central en Barcelona (Espa\~na).

\section{Ubicaci\'on y caracter\'isticas de la planta}
La planta que relevaremos se encuentra ubicada en el Centro Industrial Gar\'in, en el kil\'ometro 37 del ramal Escobar de la Ruta Panamericana.

Durante sus primeros a\~nos de actividad en el pa\'is, la empresa cont\'o con planta de producci\'on en la localidad de Munro (Buenos Aires) y oficinas comerciales en San Isidro (Buenos Aires), pero el creciente aumento de los vol\'umenes de venta fundados en la producci\'on de luminarias de avanzada tecnolog\'ia y dise\~nos, demand\'o la instalaci\'on de una planta fabril de mayor tama\~no y capacidad productiva.

IEP DE ILUMINACI\'ON cuenta desde el a\~no 2005 con instalaciones propias dentro del Centro Industrial Gar\'in (provincia de Buenos Aires), garantizando rapidez operativa y de organizaci\'on al reunir en un mismo lugar tanto sus \'areas Comerciales como las de Producci\'on y Almacenamiento. Adem\'as de esta planta, la empresa cuenta con dos filiales ubicadas en las provincias de C\'ordoba y Santa Fe. También tienen filiales en Brasil, China y México y su casa matriz se encuentra ubicada en Barcelona, España.

La empresa cuenta con su propio Laboratorio de Luminotecnia para asegurar la m\'axima calidad y seguridad en sus productos.

Brinda, adem\'as, Capacitaci\'on y seminarios a profesionales e interezados del tema.

\section{Estructura}
En la planta de Escobar se desempe\~nan cinco gerencias: Venta a Obras Privadas, Venta a Obras P\'ublicas, Administraci\'on, RR.HH. y Producci\'on.

El \'area m\'as importante es la de Producci\'on, pero la de Venta a Obras P\'ublicas tambi\'en involucra buena parte del personal de la planta. Este \'area cuenta con un Departamento de Desarrollo de Productos donde se dise\~nan y certifican productos a medida ofrecidos a dependencias p\'ublicas del \'ambito local.

\section{Productos que ofrece}
A grandes rasgos se trata de una empresa dedicada a la fabricaci\'on y comercializaci\'on de luminarias, farolas, columnas y soportes; ofreciendo soluciones de buena calidad en Alumbrado P\'ublico, \'Areas Verdes, Alumbrado Industrial, Alumbrado Interior, e incluso en Iluminaci\'on con Sumergibles y Leds.

\section{Sus clientes}
Los clientes principales de la empresa son los gobiernos municipales y provinciales de todo el territorio nacional.

Tambi\'en realizan ventan a barrios y urbanizaciones privadas y, en menor medida, a plantas industriales y cooperativas de vialidad.

\section{Sistemas de Informaci\'on}

IEP de Iluminaci\'on utiliza para la sistematizaci\'on de sus procesos un sistema integral de ERP de la empresa argentina Sistemas Bejerman, denominado eFlex.

Utilizan gran parte de los m\'odulos disponibles:

\begin{itemize}
\item Ventas y cuentas a cobrar
\item Compras y cuentas a pagar
\item Finanzas
\item Producci\'on
\item Contabilidad
\item Impuestos
\item CRM
\item Queries (Reportes)
\end{itemize}

\section{Elecci\'on de los circuitos}

En el contexto del Trabajo Pr\'actico es necesario estudiar tres circuitos de la empresa relevada. Para la selecci\'on de los mismos tenemos en cuenta diversos aspectos.

En principio tres criterios objetivos a la hora de elegir los circuitos son la buena disponibilidad de informaci\'on, fundamental para poder describirlos seg\'un los requerimientos de la c\'atedra; el soporte de los circuitos en los documentos, es decir que sean tan formales y repetibles como sea posible; y la selecci\'on de dos circuitos complejos y otro m\'as bien simple, para poder analizar circuitos interesantes pero al mismo tiempo dar a basto con la fuerza de trabajo que puede reunir el equipo.

La disponibilidad de informaci\'on est\'a en buena medida garantizada por nuestro contacto. Se trata de la Gerente de RR.HH., quien tiene conocimiento de las distintas actividades que se realizan en la empresa, y acceso a los documentos sobre las mismas. En este aspecto ya pudimos conseguir diagramas parecidos a los cursogramas que estudiamos, aunque no se atienen a normas est\'andar y que adem\'as est\'an muy fragmentados: dividen los circuitos en incumbencias muy limitadas.

Un circuito de nuestro inter\'es, por la disponibilidad de informaci\'on y su formalidad, result\'o el de Ventas. Sin embargo en la empresa el mismo tiene varias particularidades, como que se encuentra especializado para ventas privadas o p\'ublicas, que se pueden realizar con trasporte propio o externo, entre otras. Por este motivo lo consideramos muy complejo para llegar a un s\'olo cursograma de Ventas, y limitaremos el alcance del circuito estudiado.

En definitiva los circuitos elegidos ser\'an Compra de materiales y entrada de inventario, Ventas con transporte propio y Cobro de facturas.

\end{document}
